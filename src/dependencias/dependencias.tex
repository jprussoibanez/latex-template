\usepackage{dependencias/ort}

% Este es el tamaño de margen que se usa en el documento.
\usepackage[left=3cm,right=3cm,top=3cm,bottom=3cm]{geometry}

% Filter out all marginpar warnings
%\usepackage{silence}
%\WarningFilter*{latex}{Marginpar on page \thepage\space moved}

\usepackage{pdflscape}
\usepackage[spanish, es-tabla, english]{babel}
\usepackage[below]{placeins}
\usepackage{fancyhdr}
\usepackage{enumerate}
\usepackage{blindtext}
\usepackage{multicol}
\usepackage{multirow}
\usepackage{etoolbox}
\usepackage{titlesec}
\usepackage[utf8]{inputenc}
\usepackage{csquotes}
\usepackage{graphicx}
\usepackage{array}
\usepackage[table,xcdraw]{xcolor}
\usepackage{booktabs}
\usepackage[pdftex,
    pdfauthor={Estudiante 1, Estudiante 2, Estudiante 3, Estudiante 4, Estudiante 5}, % TODO
    pdftitle={Título}, % TODO
    pdfsubject={Tema}, % TODO
    pdfkeywords={Palabras clave}, % TODO
]{hyperref}
\usepackage{listings}
\usepackage{pdfpages}
\usepackage{enumitem}

% Subfigures and captions
\usepackage[figurename=Figura,tablename=Tabla]{caption}
\usepackage{subcaption}
\usepackage{xpatch}
\usepackage{hyperref}

\graphicspath{{images/}}
\titleformat{\chapter}{\normalfont\huge\bf}{\thechapter.}{20pt}{\huge}
\titlespacing*{\chapter}{0pt}{12pt}{12pt}


\makeatletter
\renewcommand{\l@section}{\@dottedtocline{1}{1.5em}{2.6em}}
\renewcommand{\l@subsection}{\@dottedtocline{2}{4.0em}{3.6em}}
\renewcommand{\l@subsubsection}{\@dottedtocline{3}{7.4em}{4.5em}}
\makeatother

\newcommand{\notAChapter}[1]{
%   \vspace*{2ex plus 1ex minus .2ex}
  \vspace*{12pt}  
  {\huge\bfseries #1}
  \vspace*{12pt}
}

\titleformat{\section}{\normalfont\LARGE\bf}{\thesection}{20pt}{\LARGE}
\titlespacing*{\section}{0pt}{12pt}{12pt}
\titleformat{\subsection}{\normalfont\Large\bf}{\thesubsection}{20pt}{\Large}
\titlespacing*{\subsection}{0pt}{12pt}{12pt}


% % % % % % % % % % % % % % % % % % % % % % % %  SUBSUBSECTION % % % % % % % % % % % % % % % % % % % % % % 
\titleformat{\subsubsection}
  {\normalfont\large\bfseries} % formato del título
  {\thesubsubsection} % número del título
  {1em} % espacio entre el número y el título
  {} % código anterior al título

% Opciones adicionales para el título
\titlespacing*{\subsubsection}{0pt}{12pt}{12pt}
% % % % % % % % % % % % % % % % % % % % % % % % % % % % % % % % % % % % % % % % % % % % % % 

% % % % % % % % % % % % % % % % % % % % % % % %  SUBSUBSUBSECTION % % % % % % % % % % % % % % % % % % % % % % 
% Definir el estilo de subsubsubsection sin afectar a paragraph
\newcommand{\subsubsubsection}[1]{%
  {\par\noindent\normalsize\textbf{#1}\par\normalsize}
}
% Ajustar el espaciado para el comando subsubsubsection similar a paragraph
\titlespacing*{\subsubsubsection}{0pt}{12pt}{12pt}

% % % % % % % % % % % % % % % % % % % % % % % % % % % % % % % % % % % % % % % % % % % % % % 

% Personalizar el estilo de paragraph y añadir un espacio después del título
\titleformat{\paragraph}[hang]
{\normalfont\normalsize\bfseries}
{\theparagraph}
{0em}
{}

\titlespacing*{\paragraph}{18pt}{12pt}{12pt}

\setcounter{tocdepth}{2} % Incluir hasta sub-subsecciones en el índice
\setcounter{secnumdepth}{3} % Numerar hasta sub-subsecciones

\renewcommand{\baselinestretch}{1.5}
\setlength{\parskip}{12pt}

\newcounter{paginas}
\setcounter{paginas}{1}
\usepackage{indentfirst}

\usepackage{color}
\definecolor{mygreen}{rgb}{0,0.6,0}
\definecolor{mygray}{rgb}{0.5,0.5,0.5}
\definecolor{mymauve}{rgb}{0.58,0,0.82}

\lstset{ %
    backgroundcolor=\color{white},
    basicstyle=\footnotesize,
    breakatwhitespace=false,
    breaklines=true,
    captionpos=b,
    literate={ó}{{\'o}}1 {ñ}{{\~n}}1,
    extendedchars=true,
    frame=single,
    keepspaces=true,
    keywordstyle=\color{blue},
    numbers=left,
    numbersep=5pt,
    numberstyle=\tiny\color{mygray},
    rulecolor=\color{mygray},
    showspaces=false,
    showstringspaces=true,
    showtabs=false,
    stepnumber=1,
    stringstyle=\color{mygreen},
    tabsize=2,
    title=\lstname
}

% Para definir colores:
% \definecolor{darkgray}{rgb}{.4,.4,.4}

\lstdefinelanguage{Javascript}{
    keywords={
        typeof, new, true, false, catch, function, return, null, catch, switch, var, if, in, while, do, else, case, break, export, private, async
    },
    keywordstyle=\color{blue}\bfseries,
    ndkeywords={class, export, boolean, throw, implements, import, this},
    ndkeywordstyle=\color{darkgray}\bfseries,
    identifierstyle=\color{black},
    sensitive=false,
    comment=[l]{//},
    morecomment=[s]{/*}{*/},
    commentstyle=\color{purple}\ttfamily,
    stringstyle=\color{red}\ttfamily,
    morestring=[b]',
    morestring=[b]"
}

\lstset{
    language=Javascript,
    extendedchars=true,
    basicstyle=\footnotesize\ttfamily,
    showstringspaces=false,
    showspaces=false,
    numbers=left,
    numberstyle=\footnotesize,
    numbersep=9pt,
    tabsize=2,
    breaklines=true,
    showtabs=false,
    captionpos=b
}

\usepackage[xindy,nonumberlist,nopostdot,style=altlist]{glossaries}
\makeglossaries

\newglossarystyle{mylist}{
    \setglossarystyle{listgroup}
    \renewenvironment{theglossary}
    {\begin{itemize}}{\end{itemize}}

    \renewcommand*{\glossentry}[2]{
        \item % bullet point
        \glstarget{##1}{\textbf{\glossentryname{##1}}:}% nombre
        \space \glossentrydesc{##1}% descripción
    }

    \renewcommand*{\glsgroupheading}[1]{
        \item[\textbf{\glsgetgrouptitle{##1}}]}
}

\usepackage[noabbrev,nameinlink,spanish]{cleveref}
\crefname{table}{tabla}{tablas}
\crefname{appendix}{Anexo}{Anexos}

\usepackage{pgfkeys}
\usepackage{longtable}

\usepackage[titletoc]{appendix}
% \titlecontents{<section>}[<left>]{<above>}
%               {<before with label>}{<before without label>}
%               {<filler and page>}[<after>]
    


\definecolor{mygreen}{rgb}{0,0.6,0}
\definecolor{mygray}{rgb}{0.5,0.5,0.5}
\definecolor{mymauve}{rgb}{0.58,0,0.82}

\lstset{ %
    backgroundcolor=\color{white},
    basicstyle=\footnotesize,
    breakatwhitespace=false,
    breaklines=true,
    captionpos=b,
    literate={ó}{{\'o}}1 {ñ}{{\~n}}1,
    extendedchars=true,
    frame=single,
    keepspaces=true,
    keywordstyle=\color{blue},
    numbers=left,
    numbersep=5pt,
    numberstyle=\tiny\color{mygray},
    rulecolor=\color{mygray},
    showspaces=false,
    showstringspaces=true,
    showtabs=false,
    stepnumber=1,
    stringstyle=\color{mygreen},
    tabsize=2,
    title=\lstname
}

\definecolor{darkgray}{rgb}{.4,.4,.4}
\definecolor{purple}{rgb}{0.65, 0.12, 0.82}
\definecolor{paleyellow}{RGB}{255, 253, 238}

\lstdefinelanguage{Javascript}{
    keywords={
        typeof, new, true, false, catch, function, return, null, catch, switch, var, if, in, while, do, else, case, break, export, private, async
    },
    keywordstyle=\color{blue}\bfseries,
    ndkeywords={class, export, boolean, throw, implements, import, this},
    ndkeywordstyle=\color{darkgray}\bfseries,
    identifierstyle=\color{black},
    sensitive=false,
    comment=[l]{//},
    morecomment=[s]{/*}{*/},
    commentstyle=\color{purple}\ttfamily,
    stringstyle=\color{red}\ttfamily,
    morestring=[b]',
    morestring=[b]"
}

\lstset{
    language=Javascript,
    extendedchars=true,
    basicstyle=\footnotesize\ttfamily,
    showstringspaces=false,
    showspaces=false,
    numbers=left,
    numberstyle=\footnotesize,
    numbersep=9pt,
    tabsize=2,
    breaklines=true,
    showtabs=false,
    captionpos=b
}

% define the key (arguments)
\pgfkeys{
    /sprint/.is family, /sprint,
    emptyParameter/.initial=,
    numero/.initial=,
    periodo/.initial=,
    objetivos/.initial=,
    planificacion/.initial=,
    cumplimiento/.initial=,
    resultados/.initial=
}

\usepackage{mathtools}
\setlength{\skip\footins}{10mm}


\newcommand{\checkmarkemoji}{\includegraphics[height=1em]{imagenes/emojis/check.png}}
\newcommand{\crossemoji}{\includegraphics[height=1em]{imagenes/emojis/cross.png}}
